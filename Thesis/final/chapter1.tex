\chapter{Introduction}
Pattern Recognition and Data Mining is a field of study that focuses on using relatively complex algorithms to discover knowledge from large pools of data. These are usually used to predict the future or to recognise patterns and label data points that are close together. The increase of computational power on the last two decades leveraged the use of techniques such as Neural Networks \cite{bishop1995neural} to tackle more complex problems such as the one of labeling digital images. The work of Lecun (1989) \cite{lecunn89} was a one of the stepping stones for all work on image recognition using Neural Networks as he was able to prove the effectiveness of stacking multiple layers of neurons to form what we call a Neural Network. These studies usually refer to how the human brain works to explain the inspiration of such techniques and each neuron was later named as perceptrons.

The efforts on the early days were focused in understanding the main principles behind human learning. For instance, recognizing handwritten digits could be seen as a trivial and effortless job for most people, however, making a computer to be able to perform this same task was not as easy as it seemed. By discovering the pattern behind digit recognition, computers would also be able to start understanding broader classes of images \cite{krizhevsky2012}. The use of computer vision techniques along with Machine Learning algorithms are nowadays the state of the art technique to overcome these challenges.

Computer Vision is a field of study that focuses on processing digital images and has Neural Networks as one of the underlying foundations for its algorithms. These are usually  focused on learning models that recognise patterns on data with several dimensions (e.g. images with width x height number of pixels). Recent advancements in both Computer Vision and Neural networks have led to the development of a new class of algorithms which is nowadays known as either Deep Learning or Deep Feed Forward Networks.

Extremely deep networks (e.g. one containing stacked perceptrons layers) are classified as Deep Learning algorithms and can be more popularly represented through deep feed forward neural networks \cite{hornik1989multilayer} and, more recently, recursive neural networks \cite{goller1996learning}. While the latter focuses on solving problems where the data points are dependent on one another (e.g. Time series) the first is more largely used on image recognition and, therefore, should be the focus of this work.

A more specific approach on feed forward nets is to extract important features from images before trying to classify them as a predefined class. This approach is widely used on a specific Neural Network architecture called Convolutional Neural Networks \cite{matsugu2003subject}. The feature extraction process can be summarized as applying specific operations in order to learn the edges of a set of images and feed these on a traditional fully connected network for classification. 

Recent work has shown that the generalization learned into those networks is rather sparse \cite{goodfellow2016}. This sparsity opens up an opportunity for methods that are able to go to unexplored data spaces in order to intentionally fool the network into predicting a different class for a given image. This operation is comprised of changing current images by adding just enough intentional noise to each pixel in order to fool an algorithm into thinking that the image has a different label \cite{goodfellow2014}\cite{papernot2016transf}\cite{goodfellow2016}\cite{szegedy2013}. The resulting images of this process are known as Adversarial Examples and their generation is done through the use of a method called Fast Gradient Sign \cite{goodfellow2014}.

This thesis presents an experimental approach on which factors could lead to less or more robustness of the Convolutional Neural Networks to Adversarial attacks. Through the use of the Fast Gradient Sign Method we will try to understand how class imbalance during training time can affect the network resistance to such attacks. Although this research focuses not only on a specific technique but also a special data space (i.e. images), the discoveries could also be generalized to other classification problems as it focuses on understanding the relationship between the algorithm capability of exploring a given data space and its overall accuracy when presented to previously unseen data points.


\section{Motivation}

As the number of dimensions in a dataset grows, the harder it is for one to visualize or explain the feature space. Therefore, exploiting vulnerabilities is one way of explaining where a specific algorithm is not performing well.The scientific understanding of such behaviors can lead to new breakthroughs on this field as the recent work of Goodfellow et al. (2014) shows on the GANs(Generative Adversarial Networks) architecture \cite{goodfellow2014generative}. 

Although the biological inspiration for neural networks comes from understanding human vision, the work from Nguyen et al.(2015) \cite{nguyen2015} have shown that unrecognizable images can be classified with pretty high confidence by DNNs. This result shows that although the technique is framed to mimic human vision, the feature space of images still needs to be explored further to avoid this kind of outcome.

Transferability in Machine Learning also shows that algorithms are vulnerable to systematic attacks as long as they are trained for the same purpose \cite{papernot2016transf}. With the ever increasing number of systems using the same techniques, one should be careful on the emerging threats being posed to their systems as several tasks are starting to heavily rely on deep learning methods.

The motivation for Adversarial robustness comes largely from being able to shield image recognition systems from behaving unexpectedly. The world is embracing Artificial Intelligence and there are already a large number of solutions that rely on these algorithms to perform tasks that range from plate recognition in car parking to general image recognition APIs such as Amazon Rekognition. In particular, differentiable machine learning algorithms in general could benefit from better understanding of gradient calculations and how they could explore the data space more efficiently.

\section{Definitions}


\section{Thesis Structure}


