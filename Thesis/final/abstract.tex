\begin{abstract}
Image recognition and machine learning have, nowadays, being used in wide variety of industries and therefore have been one of the most fast paced industries of Artificial Intelligence. A considerable amount of literature has been published on the performance of Convolutional Neural Networks and the field has evolved considerably in the last years. However, as most machine learning methods, these networks suffer from the data imbalance problem. One where the underlying training dataset is comprised of unequal number of samples for each label/class. Such difference naturally causes a phenomenon known as domain shift, which can be explained by the low generalisation capabilities of a model when presented with previously unseen data. More recent attention has focused on a technique that forces domain shift on deep networks by creating adversarial examples. These are usually comprised of small directed changes on original data points that causes inputs to be misclassified by the predictive algorithm. Recent developments in such methods have heightened the need for better of understanding of this phenomena. This study focuses on an experimental approach that sheds light on the link between the imbalanced learning problem and adversarial examples. Through a series of experiments we try to find intuitive explanations behind the success of gradient sign methods and we evaluate their effectiveness against models trained on imbalanced datasets.
\end{abstract}